%%%%%%%%%%%%%%%%%%%%%%%%%%%%%%%%%%%%%%%%%
% Tufte-Style Book (Minimal Template)
% LaTeX Template
% Version 1.0 (5/1/13)
%
% This template has been downloaded from:
% http://www.LaTeXTemplates.com
%
% License:
% CC BY-NC-SA 3.0 (http://creativecommons.org/licenses/by-nc-sa/3.0/)
%
% IMPORTANT NOTE:
% In addition to running BibTeX to compile the reference list from the .bib
% file, you will need to run MakeIndex to compile the index at the end of the
% document.
%
%%%%%%%%%%%%%%%%%%%%%%%%%%%%%%%%%%%%%%%%%

%----------------------------------------------------------------------------------------
%	PACKAGES AND OTHER DOCUMENT CONFIGURATIONS
%----------------------------------------------------------------------------------------

\documentclass{tufte-book} % Use the tufte-book class which in turn uses the tufte-common class

\hypersetup{colorlinks} % Comment this line if you don't wish to have colored links

\usepackage{microtype} % Improves character and word spacing

\usepackage{lipsum} % Inserts dummy text

\usepackage{booktabs} % Better horizontal rules in tables

\usepackage{listings}

\usepackage{graphicx} % Needed to insert images into the document
\graphicspath{{graphics/}} % Sets the default location of pictures
\setkeys{Gin}{width=\linewidth,totalheight=\textheight,keepaspectratio} % Improves figure scaling

\usepackage{fancyvrb} % Allows customization of verbatim environments
\fvset{fontsize=\normalsize} % The font size of all verbatim text can be changed here

\newcommand{\hangp}[1]{\makebox[0pt][r]{(}#1\makebox[0pt][l]{)}} % New command to create parentheses around text in tables which take up no horizontal space - this improves column spacing
\newcommand{\hangstar}{\makebox[0pt][l]{*}} % New command to create asterisks in tables which take up no horizontal space - this improves column spacing

\usepackage{xspace} % Used for printing a trailing space better than using a tilde (~) using the \xspace command

\newcommand{\monthyear}{\ifcase\month\or January\or February\or March\or April\or May\or June\or July\or August\or September\or October\or November\or December\fi\space\number\year} % A command to print the current month and year

\newcommand{\openepigraph}[2]{ % This block sets up a command for printing an epigraph with 2 arguments - the quote and the author
\begin{fullwidth}
\sffamily\large
\begin{doublespace}
\noindent\allcaps{#1}\\ % The quote
\noindent\allcaps{#2} % The author
\end{doublespace}
\end{fullwidth}
}

\newcommand{\blankpage}{\newpage\hbox{}\thispagestyle{empty}\newpage} % Command to insert a blank page

\usepackage{makeidx} % Used to generate the index
\makeindex % Generate the index which is printed at the end of the document

%----------------------------------------------------------------------------------------
%	BOOK META-INFORMATION
%----------------------------------------------------------------------------------------

\title{S54MTB Mini Keyer} % Title of the book

\author{S54MTB} % Author

\publisher{Mare and Gal Electronics} % Publisher

%----------------------------------------------------------------------------------------

\begin{document}

\frontmatter

%%%----------------------------------------------------------------------------------------
%%%	EPIGRAPH
%%%----------------------------------------------------------------------------------------
%%
%%\thispagestyle{empty}
%%\openepigraph{Quotation 1}{Author, {\itshape Source}}
%%\vfill
%%\openepigraph{Quotation 2}{Author}
%%\vfill
%%\openepigraph{Quotation 3}{Author}

%----------------------------------------------------------------------------------------

\maketitle % Print the title page

%----------------------------------------------------------------------------------------
%	COPYRIGHT PAGE
%----------------------------------------------------------------------------------------

\newpage
\begin{fullwidth}
~\vfill
\thispagestyle{empty}
\setlength{\parindent}{0pt}
\setlength{\parskip}{\baselineskip}
Copyright \copyright\ \the\year\ \thanklessauthor

\par\smallcaps{Published by \thanklesspublisher}

\par\smallcaps{\url{http://e.pavlin.si}}

\par Everyone is permitted to copy and distribute verbatim copies of this document, but changing it is not allowed.\index{license}

\par\textit{V1.0, rev.0.a, \monthyear}
\end{fullwidth}

%----------------------------------------------------------------------------------------

\tableofcontents % Print the table of contents

%%----------------------------------------------------------------------------------------
%
%\listoffigures % Print a list of figures
%
%%----------------------------------------------------------------------------------------
%
%\listoftables % Print a list of tables
%
%%----------------------------------------------------------------------------------------
%%	DEDICATION PAGE
%%----------------------------------------------------------------------------------------
%
%\cleardoublepage
%~\vfill
%\begin{doublespace}
%\noindent\fontsize{18}{22}\selectfont\itshape
%\nohyphenation
%Dedicated to my family and friends.
%\end{doublespace}
%\vfill
%\vfill

%----------------------------------------------------------------------------------------
%	INTRODUCTION
%----------------------------------------------------------------------------------------

\cleardoublepage
\chapter*{Introduction} % The asterisk leaves out this chapter from the table of contents

%Citation example \cite{Tufte2001}, notice how the citation is in the margin. This is an example of how to add something to the index at the end of the document.\index{citation}
%
%\newthought{Example of} the \texttt{newthought} command for starting new sections. Typography examples: \allcaps{all caps} and \smallcaps{small caps}.

%------------------------------------------------
\begin{fullwidth}
	
An "electronic keyer" generates the "dits and dahs" of the International Morse Code when properly configured. This manual will guide you through the setup process and describe all instructions available. Once keyer is prepared it requires only 5V to operate. 
\end{fullwidth}
	
\section{Conventions}

This guide uses commonly used conventions for command syntax and displayed information. Command Syntax Statements may include:
\begin{itemize}
	\item Vertical bars (|) represent alternative, mutually exclusive elements.
	\item Square brackets ([]) indicate optional elements.
	\item Braces (<>) enclose required elements.
\end{itemize}




%%\section{Figures}
%%
%%\lipsum[1] 
%%
%%\begin{marginfigure}
%%\includegraphics[width=\linewidth]{helix}
%%\caption{This is a margin figure. The helix is defined by $x = \cos(2\pi z)$, $y = \sin(2\pi z)$, and $z = [0, 2.7]$. The figure was drawn using Asymptote (\url{http://asymptote.sf.net/}).}
%%\label{fig:marginfig}
%%\end{marginfigure}
%%
%%\lipsum[2]
%%
%%\begin{figure*}[h]
%%\includegraphics[width=\linewidth]{sine.pdf}
%%\caption{This graph shows $y = \sin x$ from about $x = [-10, 10]$.
%%\emph{Notice that this figure takes up the full page width.}}
%%\label{fig:fullfig}
%%\end{figure*}
%%
%%\lipsum[3]

%------------------------------------------------

%%\section{Tables} \marginnote{This is a random margin note. Notice that there isn't a number preceding the note, and there is no number in the main text where this note was written. Use \texttt{sidenote} to use a number.}
%%
%%\lipsum[4]
%%
%%\begin{table} % Add the following just after the closing bracket on this line to specify a position for the table on the page: [h], [t], [b] or [p] - these mean: here, top, bottom and on a separate page, respectively
%%\centering % Centers the table on the page, comment out to left-justify
%%\begin{tabular}{l c c c c c} % The final bracket specifies the number of columns in the table along with left and right borders which are specified using vertical bars (|); each column can be left, right or center-justified using l, r or c. To specify a precise width, use p{width}, e.g. p{5cm}
%%\toprule % Top horizontal line
%%& \multicolumn{5}{c}{Growth Media} \\ % Amalgamating several columns into one cell is done using the \multicolumn command as seen on this line
%%\cmidrule(l){2-6} % Horizontal line spanning less than the full width of the table - you can add (r) or (l) just before the opening curly bracket to shorten the rule on the left or right side
%%Strain & 1 & 2 & 3 & 4 & 5\\ % Column names row
%%\midrule % In-table horizontal line
%%GDS1002 & 0.962 & 0.821 & 0.356 & 0.682 & 0.801\\ % Content row 1
%%NWN652 & 0.981 & 0.891 & 0.527 & 0.574 & 0.984\\ % Content row 2
%%PPD234 & 0.915 & 0.936 & 0.491 & 0.276 & 0.965\\ % Content row 3
%%JSB126 & 0.828 & 0.827 & 0.528 & 0.518 & 0.926\\ % Content row 4
%%JSB724 & 0.916 & 0.933 & 0.482 & 0.644 & 0.937\\ % Content row 5
%%\midrule % In-table horizontal line
%%\midrule % In-table horizontal line
%%Average Rate & 0.920 & 0.882 & 0.477 & 0.539 & 0.923\\ % Summary/total row
%%\bottomrule % Bottom horizontal line
%%\end{tabular}
%%\caption{Table caption text} % Table caption, can be commented out if no caption is required
%%\label{tab:template} % A label for referencing this table elsewhere, references are used in text as \ref{label}
%%\end{table}

%----------------------------------------------------------------------------------------

\mainmatter


%----------------------------------------------------------------------------------------
%	CHAPTER 0 - Messages
%----------------------------------------------------------------------------------------

\chapter{Preparation}
\label{ch:0}
\section{Virtual COM Port Driver}
When device is plugged into USB port for the first time, the system will try to install STM32 Virtual COM Port Driver. 
\marginnote{Don't plug the device into USB port before installing the driver.}
The driver can be downloaded from STM web page \url{http://www.st.com/web/en/catalog/tools/PF257938#}. ST claims to support following x86 \& x64 Windows platforms: Windows 98SE, 2000, XP, Vista, Seven, 8.x. 

After downloading, install the VCP Driver:
\begin{enumerate}
	\item Uninstall previous versions (Start-> Settings-> Control Panel-> Add or remove programs)
	\item Run your "VCP\_V1.4.0\_Setup.exe"
	\item Go to Your installation directory - Example, C:\\Program Files (x86)\\STMicroelectronics\\Software\\Virtual comport driver
	\item Go to Your OS version directory ([Win7] or [Win8])
\end{enumerate}
   and finally:      
\begin{itemize}
	\item Double click on dpinst\_x86.exe on a 32-bits OS version
	\item Double click on dpinst\_amd64.exe on a 64-bits OS version
\end{itemize}      
...and follow the instructions.

After installation plug the module into one of the USB ports. 

\section{Terminal}
Terminal is used to communicate with the Mini Keyer. 



%----------------------------------------------------------------------------------------
%	CHAPTER 1 - Messages
%----------------------------------------------------------------------------------------

\chapter{Messages}
\label{ch:1}

%------------------------------------------------

\section{MSG}


\section{MODE}


\section{RPT}



\section{Section 1 - Fullwidth Environment Example}

\begin{fullwidth}
\lipsum[5]
\end{fullwidth}

\subsection{Subsection 1}




\begin{lstlisting}[language=Python, caption=Python example]
	import numpy as np
	
	def incmatrix(genl1,genl2):
	m = len(genl1)
	n = len(genl2)
	M = None #to become the incidence matrix
	VT = np.zeros((n*m,1), int)  #dummy variable
	
	#compute the bitwise xor matrix
	M1 = bitxormatrix(genl1)
	M2 = np.triu(bitxormatrix(genl2),1) 
	
	for i in range(m-1):
	for j in range(i+1, m):
	[r,c] = np.where(M2 == M1[i,j])
	for k in range(len(r)):
	VT[(i)*n + r[k]] = 1;
	VT[(i)*n + c[k]] = 1;
	VT[(j)*n + r[k]] = 1;
	VT[(j)*n + c[k]] = 1;
	
	if M is None:
	M = np.copy(VT)
	else:
	M = np.concatenate((M, VT), 1)
	
	VT = np.zeros((n*m,1), int)
	
	return M
\end{lstlisting}





\subsection{Subsection 2}

\lipsum[7-8]

%------------------------------------------------

\section{Section 2}

\subsection{Subsection 1}

\lipsum[9-10]

\subsection{Subsection 2}

\lipsum[11-12]

%----------------------------------------------------------------------------------------
%	CHAPTER 2 - Audio
%----------------------------------------------------------------------------------------

\chapter{Audio}
\label{ch:2}

\section{AUD}


%----------------------------------------------------------------------------------------
%	CHAPTER 3 - Output
%----------------------------------------------------------------------------------------

\chapter{Output}
\label{ch:3}

\section{OUT}



%----------------------------------------------------------------------------------------
%	CHAPTER 4 - Storage
%----------------------------------------------------------------------------------------

\chapter{Storage}
\label{ch:4}
\subsection{CAT}
\subsection{STORE}
\subsection{LOAD}
\subsection{DEL}
\subsection{ID}



%----------------------------------------------------------------------------------------
%	CHAPTER 5 - Running 
%----------------------------------------------------------------------------------------

\chapter{Running}
\label{ch:5}
\subsection{RUN}

\subsection{STOP}


%%%----------------------------------------------------------------------------------------
%%
%%\backmatter
%%
%%%----------------------------------------------------------------------------------------
%%%	BIBLIOGRAPHY
%%%----------------------------------------------------------------------------------------
%%
%%\bibliography{bibliography} % Use the bibliography.bib file for the bibliography
%%\bibliographystyle{plainnat} % Use the plainnat style of referencing
%%
%%%----------------------------------------------------------------------------------------
%%
%%\printindex % Print the index at the very end of the document

\end{document}